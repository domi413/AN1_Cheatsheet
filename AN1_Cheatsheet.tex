% Document
\documentclass{article}
\usepackage[utf8]{inputenc}
\usepackage[margin=1.6cm]{geometry}
\usepackage{multicol}
\usepackage{hyperref}

% Math
\usepackage{amsmath}

% Configuration
\pagestyle{empty} % To remove page enumeration
\renewcommand{\baselinestretch}{0.9}

\hypersetup{
	pdftitle={AN1-Cheatsheet},
	pdfauthor={Dominik},
	pdfsubject={AN1},
}

\begin{document}
\begin{multicols}{2}

\section*{Produktregel}

\(
	f(x) = u(x) \cdot v(x) \\
	f'(x) = u'(x) \cdot v(x) + u(x) \cdot v'(x)
\)

\section*{Quotientenregel}

\(
	f(x) = \dfrac{g(x)}{h(x)} \\
	f'(x) = \dfrac{h(x) \cdot g'(x) - h'(x) \cdot g(x)}{h(x)^2}
\)

\section*{Kettenregel}

\(
	f(x) = u(v(x)) \\
	f'(x) = u'(v(x)) \cdot v'(x)
\)

\section*{Stetigkeit beweisen}

\begin{minipage}{0.4\textwidth}
	Wenn zwei Funktionen, die in unterschiedlichen Bereichen definiert sind, an einem bestimmten Punkt stetig sein sollen, kann man dies Gleichsetzen. \\
	
	Für eine Funktion \( f(x) \), die definiert ist als:
	\[
		f(x) = 
		\begin{cases} 
			g(x) & \text{für } x < c \\
			h(x) & \text{für } x > c 
		\end{cases}
	\]
	wobei \( c \) der Punkt ist, an dem die Funktion stetig sein soll, muss gelten:
	\[
		\lim_{{x \to c^-}} g(x) = \lim_{{x \to c^+}} h(x)
	\]
\end{minipage}

\section*{Linearisierung}

Linearisierung beim Punkt p:

\( \\
	f(x) = g(x) \\
	r(x) = f(x) - g(x) \\
	r'(x) \text{ berechnen} \\
	x1 = p - \dfrac{r(p)}{r'(p)}
\)

\section*{Extremalstellen}

\(
	f(x) \\
	f'(x) \text{ berechnen} \\
	f'(x) = 0 \ \rightarrow \ \text{z.B. } \dfrac{3x^2-3}{1+x^2} = 0 \\
	\rightarrow \text{Kritische Punkte} \\
	\text{(Der Nenner muss nur berücksichtigt} \\
	\text{werden, wenn er 0 werden kann)} \\\\
	f''(x) \text{ berechnen} \\\\
	\text{Kritische Punkte p1, p2 einsetzen:} \\
	f''(p1) > 0 \ \rightarrow \text{lokales Minimum} \\
	f''(p2) < 0 \ \rightarrow \text{lokales Maximum}
\)

\section*{Ableitungen}

\begin{tabular}{l|l}
	\textbf{Funktion}           & \textbf{Ableitungsfunktion}              \\ \hline
	\( x \mapsto x^a \)         & \( x \mapsto a \cdot x^{a-1} \)          \\
	\( x \mapsto 1 \)           & \( x \mapsto 0 \)                        \\
	\( x \mapsto x \)           & \( x \mapsto 1 \)                        \\
	\( x \mapsto x^2 \)         & \( x \mapsto 2x \)                       \\
	\( x \mapsto \frac{1}{x} \) & \( x \mapsto -\frac{1}{x^2} \)           \\
	\( x \mapsto \sqrt{x} \)    & \( x \mapsto \frac{1}{2\sqrt{x}} \)      \\
	\( x \mapsto e^x \)         & \( x \mapsto e^x \)                      \\
	\( x \mapsto a^x \)         & \( x \mapsto \ln(a) \cdot a^x \)         \\
	\( x \mapsto \ln(x) \)      & \( x \mapsto \frac{1}{x} \)              \\
	\( x \mapsto \log_b(x) \)   & \( x \mapsto \frac{1}{\ln(b) \cdot x} \) \\
	\( x \mapsto \sin(x) \)     & \( x \mapsto \cos(x) \)                  \\
	\( x \mapsto \cos(x) \)     & \( x \mapsto -\sin(x) \)                 \\
	\( x \mapsto \tan(x) \)     & \( x \mapsto \frac{1}{\cos^2(x)} \)      \\
	\( x \mapsto \tan(x) \)     & \( x \mapsto 1 + \tan^2(x) \)            \\
	\( x \mapsto \arcsin(x) \)  & \( x \mapsto \frac{1}{\sqrt{1-x^2}} \)   \\
	\( x \mapsto \arccos(x) \)  & \( x \mapsto -\frac{1}{\sqrt{1-x^2}} \)  \\
	\( x \mapsto \arctan(x) \)  & \( x \mapsto \frac{1}{1+x^2} \)
\end{tabular}

\section*{Potenzgesetze}

\begin{flalign*}
	a^n a^m &= a^{n+m} \\
	\frac{a^n}{a^m} &= a^{n-m} \\
	a^n b^n &= (ab)^n \\
	\frac{a^n}{b^n} &= \left(\frac{a}{b}\right)^n \\
	a^{n \cdot m} &= (a^n)^m \\
	a^{-n} &= \frac{1}{a^n} \\
	a^0 &= 1 \\
	a^1 &= a
\end{flalign*}

\section*{Logarithmusgesetze}

\begin{align*}
	a^{\log_a(x)} &= x \\
	\log_a (x \cdot y) &= \log_a (x) + \log_a (y) \\
	\log_a \left( \frac{x}{y} \right) &= \log_a (x) - \log_a (y) \\
	\log_a (x^b) &= b \cdot \log_a (x)
\end{align*}

\section*{Sinus, Kosinus, Tangens}

\begin{align*}
	\tan(x) &= \frac{\sin(x)}{\cos(x)} \\
	1 &= \cos^2(x) + \sin^2(x)
\end{align*}

\end{multicols}
\end{document}
